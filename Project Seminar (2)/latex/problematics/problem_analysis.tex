\chapter{Problem analysis} \label{chap:problemanalysis}

% nejaky obkec o tom ze data sa ziskavaju cez teleskopy a tie maju CCD chipy a ze snimky su vo formate FITS. a potom ze v tejto casti prejdeme rozne features ktore sa na snimkach nachadzaju, scenare a defekty a ako tieto nezaduce data odstranujeme
% FITS frames
% AGO data

Many of the most interesting dynamics in nature have to do with interactions between organisms. These interactions are often subtle, indirect, and difficult to detect. Interactions in which one organism consumes all or part of another. This includes predator-prey, herbivore-plant, and parasite-host interactions. These linkages are the prime movers of energy through food chains. They are an important factor in the ecology of populations, determining the mortality of prey and the birth of new predators.

A connected system of predator and prey should cycle, according to mathematical models and common sense: predators grow in number when prey is abundant, predation reduces prey populations to low levels, predator numbers fall, and prey populations rise indefinitely. The Lotka-Volterra Model is one such model that mimics predator-prey interactions.

\section{Classic version of predator-prey game}


The classical predator-prey model is a mathematical model used to study the dynamics of populations in a predator-prey ecosystem. The model typically consists of a set of ordinary differential equations that describe the change in the population of two species, such as predators and prey, over time.

The model considers the rate of change in the prey population as a function of their own birth and death rates, and the rate of predation by the predators. Similarly, the rate of change in the predator population is a function of their own birth and death rates, and the rate of prey consumed.

The model assumes that the populations of both species are continuously growing or declining depending on the balance of birth and death rates, and the interaction between the two species. In particular, the prey population grows when there are few predators and declines when there are many, and the predator population declines when there is little prey and grows when there is plenty.

This model helps to understand the underlying mechanisms driving the fluctuations in populations of both species, and can be used to make predictions about future population dynamics based on changes in environmental conditions or other factors.

The Lotka-Volterra equations describe an ecological predator-prey (or parasite-host) model which assumes that, for a set of fixed positive constants \(a\) (the growth rate of prey), \(b\) (the rate at which predators destroy prey), \(r\) (the death rate of predators), and \(c\) (the rate at which predators increase by consuming prey). The following conditions will be held for our computer simulation model.

Let the prey population at time t be given by \(y_1(t)\), and the predator population by \(y_2(t)\). Assume that, in the absence of predators, the prey will grow exponentially according to \(y_1' = a \ y_1\) for a certain \(a > 0\). We also assume that the death rate of the prey due to interaction is proportional to \(y_1(t) \ y_2(t)\), with a positive proportionality constant. So:
\(\\ y_1' (t) = a\ y_1(t) - b\ y_1(t)\ y_2(t)   \)

Without prey, predators will die exponentially according to \(y_2'(t)=-r\ y_2\ dt\) for a certain \(r>0\). Their birth strongly depends on both population sizes, so we finally find for a certain \(c>0\):
\( \\ y_2'(t) = -r\ y_2(t) + c\ y_1(t)\ y_2(t)  \)

These equations lead to the following system of differential equations:  

$$ \left\{
\begin{array}{lr}
y_1'(t) = a\ y_1(t) - b\ y_1(t)\ y_2(t)\\
y_2'(t) = -r\ y_2(t) + c\ y_1(t)\ y_2(t)
\end{array}
\right. $$
\newline
 
We see that both \(( e^{at}, 0)\) and \((0, e^{-ct} )\) are solutions of \((y_1(t), y_2(t))\). From this system, we find that for every solution we must have 
\(y_1' (\frac{r}{y_1} - c) +y_2' (\frac{a}{y_2} - b) = 0 \).

Integrating both sides give us: 
\[r\ log\ y_1(t) -   c\ y_1(t)  +  a\ log\ y_2(t)\ - \ b\ y_2(t) = constant. \]

The Lotka-Volterra equations are a pair of first-order, non-linear, differential equations that describe the dynamics of biological systems in which two species interact. The earliest predator-prey model based on sound mathematical principles forms the basis of many models used today in the analysis of population dynamics, the original form has problems \cite{cpp1}.

The game is usually represented by a payoff matrix, where each entry represents the payoffs (or utilities) to the predator and prey given their current strategies. The strategies in this game correspond to the population sizes of the predator and prey, and the payoffs are determined by the underlying population dynamics described by the predator-prey model.

For example, suppose the prey population grows at a rate proportional to its size and the predator population grows at a rate proportional to the number of prey consumed. Then, the payoffs to the predator and prey can be represented as follows:


Predator Payoff = \((a - b) * P - c * Q\),

Prey Payoff = \(d * Q - e * P\),


where \(P\) and \(Q\) are the population sizes of the predator and prey, respectively, \(a, b, c, d,\) and \(e\) are positive constants representing the growth and death rates of the populations, and the terms \((a - b) * P\) and \(d * Q\) represent the intrinsic growth rates of the populations. The terms \(-c * Q\) and \(-e * P\) represent the costs of predation and being preyed upon, respectively.

This game has a unique Nash equilibrium, which is a state where neither player has an incentive to change its strategy, given the strategies of the other player. At the Nash equilibrium, the population sizes of both species are stable and remain constant over time.

The classical predator-prey game provides a framework for analyzing the interplay between the populations of predators and prey in an ecosystem, and can be used to understand the factors that influence the stability and persistence of both species.

\input{problematics/2.2modifications}
\section{Introduction to the fuzzy logic}


Fuzzy logic, a mathematical theory for uncertainty, was introduced by Zadeh in 1965 as an extension of boolean logic, based on the mathematical theory of fuzzy sets. Fuzzy logic allows for a condition to be in a state other than just true or false by introducing the concept of degree in the verification process. This enables the consideration of inaccuracies and uncertainties, giving reasoning a valuable degree of flexibility. One of the many benefits of fuzzy logic is its ability to formalize human reasoning through natural language-based rules. Fuzzy systems are designed to emulate human reasoning processes, and if-then rules are used to represent relationships explicitly.

\subsection{Fuzzy sets}

Fuzzy sets generalize the classical concept of a set. Their motivation stems from the following idea: If somebody wants us to specify a set of all speed values which can be referred as a high speed value. First we can say that the speed range for what we can specify as a high speed is between 130 and 480. So, we start with the set \(U = [130, 480]\) (kph). In further reasoning, however, we will be confronted with insurmountable difficulties,
namely, we find out that we are not able to specify high speed value precisely. For example,
if we decide that the high speed is at least 140, then we can
immediately ask: “What about the speed 139 kph?” We are not able to distinguish
such speed difference by naked eye only. But, according to our decision,
the first value is considering as a high speed value, and the second one is not. We arrive at clearly counterintuitive
conclusion. Therefore, in a common practice, we cannot understand numbers to
be fully precise but rather to be imprecise, for example, “approximately 150”. Alternatively,
we can use vague linguistic expressions such as “slow” and “very fast”. \cite{NOPEDV16}
The general definition of a fuzzy set is the following.

Let \(U\) be a set called universe. A fuzzy set is a function
\[A : U \rightarrow [0, 1].\]

This function is also called a membership function of the fuzzy set A. If \(u \in \ U\), then the number \(A(u) \in [0, 1]\) is called a membership degree of \(u\) in the fuzzy set \(A\).

\subsection{Fuzzy operators}

The notation and operations of fuzzy logic are based on classical logic and propositional calculus, a modern form of classical logic notation. In classical logic, propositions are either true or false, with nothing in between. It is often common to assign numerical values to the truth of propositions, where 1 means true and 0 means false. An important principle of classical logic is the Law of Excluded Middle, which states that a proposition must be either true or false, and the Law of Non-contradiction, which states that a proposition cannot be both true and false. A statement cannot be true and false at the same time.

The truth value of complex propositions is obtained by combining the truth values
of the elemental propositions, which enter into the complex proposition. The most
common operators are NOT, AND (A AND B is true if both A
and B are true) and OR (A OR B is true if either A or B or both are true.)\cite{FesFr}

The evaluation of formulae A AND B and A OR B is shown in Table \ref{table:andor}, a presentation is called truth table.

\begin{table}[ht]
\caption{Truth table for AND and OR logic operators}
\centering
\begin{tabular}{c c c c}    \toprule
\emph{A} & \emph{B} &\emph{A AND B}& \emph{A OR B}   \\\midrule
0    & 0  & 0  & 0  \\ 
0  &1 &0 & 1\\ 
1 & 0 & 0 & 1\\
1  & 1 & 1 & 1 \\
 \hline
\end{tabular}
\label{table:andor}
\end{table}

For all fuzzy operators, 
\[NOT\  A = 1-A\]

\(A \ AND \ B\)

Zadeh operator: \(A \  AND \  B = min(A, \  B)\)

Probabilistic operator, assuming independance: \( A \  AND \  B = A \ast B\)

Bounded difference operator: \(A\  AND \  B = max(0, A+B-1)\)

\(A \  OR \  B\)

Zadeh operator: \(A \  OR \  B = max(A, B)\)

Probabilistic operator, assuming independance: \( A \  OR \  B = A + B - A \ast B\)

Bounded difference operator: \(A \  OR \  B = min(1, A+B)\)


\subsection{Fuzzification}

Fuzzification means to find grades of membership of linguistic values of a linguistic
variable corresponding to an input number, scalar or fuzzy.



\subsection{Defuzzification}

At the end of an advanced rule-firing sequence in a fuzzy expert system, a fuzzy conclusion \(C\) is often reached. However, \(C\) is a linguistic variable with graded membership values, and we usually need to compute a single scalar that corresponds to these values. This is where defuzzification comes in – the process of converting \(C\) into a scalar that can be sent as a signal to the process in question, especially in fuzzy control.
Defuzzification is more intricate than fuzzification, with multiple choices to be made and many methods proposed. In this article, we will highlight the essential areas where choices must be made and identify the most commonly used choices, rather than exploring every possibility.
Assuming we know the grades of membership of the fuzzy set to be defuzzified, the first step in defuzzification is to decide how to modify the membership functions for linguistic values to reflect that each value probably has a different grade of membership. This modification involves ANDing the membership function \(m(x,value)\) with \(m(value,lvar)\), where \(lvar\) is the linguistic variable of which value is a member, \(m(x,value)\) is the membership of real number x in value, and \(m(value,lvar)\) is the membership of value in \(lvar\). The result of the modification is a new membership function for value called \(m0(x,value)\).

The most common choices for the AND operator are the Zadehian
min(A, B), often known as the Mamdani method because of its early successful
use in process control by Mamdani (1976). 

Next, the individual membership functions must be aggregated
into a single membership function for the entire linguistic variable. Aggregation
operators resemble t-conorms, but with fewer restrictions, the Zadehian max OR operator is frequently used.

In the last step, we find a single number compatible with the membership function
and this number will be the output from this final step in the
defuzzification process.\cite{FesFr}

The fuzzy set has a following form \(A =\{a_1 / u_1, ... , a_r / u_r \} \) defined on a finite universe \(U\).\cite{NOPEDV16}

Center of Gravity/Area defuzzification is the most often used method and is used for fuzzy approximation problems:
\[COG(A)=\frac{{\sum_{k=1}^r {A(u_k)\cdot u_k}}}{\sum_{k=1}^r {A(u_k)}}\]

Mean of maxima defuzzification is computationally simpler than COG:

\[MOM(A)=\frac{1}{r_{max}}\sum_{j=1}^{r_{max}} {u_j^{max}}\]

First of maxima and last of maxima defuzzifications are the simplest defuzzification methods:

\[FOM(A) = min\{u_j^{max}| j= 1, ...,r_{max}\},\]

\[LOM(A) = max\{u_j^{max}| j= 1, ...,r_{max}\}.\]

Center of sums defuzzification is a variant of COG. Let us assume that set a is a union of fuzzy sets
\[A=B_1\cup ... \cup B_s .\]
\[COS(A)=\frac{\sum_{j=1}^{s} {(\sum_{k=1}^{r} {u_k\cdot B_j (u_k)})}}{\sum_{j=1}^{s} {\sum_{k=1}^{r} {B_j (u_k)}}}.\]

DEE fuzzification, it classifies a fuzzy set to
be defuzzified into one of the three types—\(Z\), \(M\), and \(S\):

\[
  DEE(A)=
  \begin{cases}
    LOM(A) &\text{if $A$ is of type $Z$} \\
    MOM(A) &\text{if $A$ is of type $M$}\\
    FOM(A) &\text{if $A$ is of type $S$}.
  \end{cases}
\]




\section{Adaptive neuro‑fuzzy inference system (ANFIS)}
An adaptive network is a multi-layer feedforward network in which each node performs a particular
function (node function) on incoming signals as well as a set of parameters pertaining to this node. The nature
of the node functions may vary from node to node, and the choice of each node function depends on the overall
input-output function which the adaptive network is required to carry out.

ANFIS represent Sugeno-Takagi fuzzy models and uses a hybrid learning algorithm. It has five layers as
shown in figure \ref{img:archit}.

The first hidden layer is
responsible for the mapping of the input variable
relatively to each membership functions. The
operator T-norm is applied in the second hidden
layer to calculate the antecedents of the rules. The
third hidden layer normalizes the rules strengths
followed by the fourth hidden layer where the
consequents of the rules are determined. The output
layer calculates the global output as the summation
of all the signals that arrive to this layer.
ANFIS uses backpropagation learning to determine
the input membership functions parameters and the
least mean square method to determine the
consequents parameters. Each step of the iterative
learning algorithm has two parts. In the first part, the
input patterns are propagated and the parameters of
the consequents are calculated using the iterative
minimum squared method algorithm, while the
parameters of the premises are considered fixed. In
the second part, the input patterns are propagated
again and in each iteration, the learning algorithm
backpropagation is used to modify the parameters of
the premises, while the consequents remain fixed. \cite{inproceedings}

\begin{figure}[h]
    \centering
    \includegraphics[width=.9\textwidth]{problematics/figure2}
    \caption{ANFIS architecture with two inputs, one outpur and four rules}
    \label{img:archit}
\end{figure}

\section{Multiagent system}


An agent is an autonomous computer system that functions on behalf of its owner or user. Unlike being told what to do, an agent can independently determine what actions it must take to meet its objectives. A multiagent system comprises several agents that interact with each other by exchanging messages through a network infrastructure. In the case of multiagent systems, each agent represents or acts on behalf of users or owners who may have different goals and motivations. Therefore, to interact successfully, agents must possess the necessary skills to collaborate, coordinate, and negotiate with each other, much like how we interact with people in our daily lives.

Multiagent systems are - by definition - a subclass of concurrent
systems, and there are some in the distributed systems community who
question whether multiagent systems are sufficiently different to 'standard' distributed/
concurrent systems to merit separate study.

In multiagent systems, however, there are two important twists to the concur -
rent systems story. \cite{MultiAgent}
\begin{itemize}
  \item  First, because agents are assumed to be autonomous - capable of making
independent decisions about what to do in order to satisfy their design
objectives - it is generally assumed that the synchronization and coordination
structures in a multiagent system are not hardwired in at design
time, as they typically are in standard concurrent/distributed systems.
We therefore need mechanisms that will allow agents to synchronize and
coordinate their activities at run time.
  \item  Second, the encounters that occur among computing elements in a multiagent
system are economic encounters, in the sense that they are encounters
between self-interested entities. In a classic distributed/concurrent system,
all the computing elements are implicitly assumed to share a common goal
(of making the overall system function correctly). In multiagent systems, it is
assumed instead that agents are primarily concerned with their own welfare
(although of course they will be acting on behalf of some user/owner).
\end{itemize}

\section{Current overview of fuzzy modeling predator-prey game} \label{sec:scenarios}

Scientists use predator-prey models to study how animals that hunt (predators) and animals that are hunted (prey) interact in a certain environment. Fuzzy modeling is a way to study complicated things using fuzzy logic.

In a fuzzy game, the population dynamics of predators and prey are modeled using fuzzy logic. Fuzzy logic helps us understand better how predators and prey relate to each other. It does this by considering the fact that ecological systems are often uncertain and imprecise.

You can use a fuzzy inference system to model a game where a fuzzy predator chases a fuzzy prey. It has three parts: making things fuzzy, figuring out what to do, and making things clear again. Fuzzification is when we turn simple information like how many people live somewhere or things in the environment into fuzzy groups. Inference means using fuzzy rules to make predictions about how predators and prey populations will change. Defuzzification means changing the unclear results back into clear data.

Fuzzy modeling has been applied to predator-prey games in various ways to understand the behavior of the agents and to develop optimal strategies for managing populations in real-world scenarios. Here are some current overviews of fuzzy modeling predator-prey games:

Fuzzy logic-based predator-prey models: These models use fuzzy logic to develop rules for the agents' behavior based on their characteristics and the environment. The rules are defined using linguistic terms and fuzzy sets that represent the degree of membership of an agent in a particular category. These models have been used to simulate different scenarios and to evaluate the effectiveness of different strategies for managing predator and prey populations.

Fuzzy game theory-based predator-prey models: These models use fuzzy logic and game theory to analyze the behavior of the agents in predator-prey games. The models are based on the assumption that both predators and prey have incomplete information about the behavior of the other agents and are uncertain about the outcome of their actions. The models use fuzzy logic to represent the uncertainty and game theory to analyze the interactions between the agents.

Fuzzy reinforcement learning-based predator-prey models: These models use fuzzy logic and reinforcement learning to develop optimal strategies for the agents in predator-prey games. The models use fuzzy logic to represent the state of the environment and the actions of the agents and use reinforcement learning algorithms to learn the best actions to take in different situations. These models have been used to develop effective strategies for managing predator and prey populations in real-world scenarios.

Overall, fuzzy modeling can provide a more comprehensive and accurate representation of the complex dynamics between predators and prey in ecological systems. It can also be used to explore different scenarios and predict the outcomes of different management strategies.

\input{problematics/2.7review}
\section{Matlab Fuzzy Logic Toolbox}

Fuzzy Logic Toolbox provides a range of useful features for MATLAB users, including functions, apps, and a Simulink block for analyzing, designing, and simulating fuzzy logic systems. The product enables you to configure inputs, outputs, membership functions, and rules of both type-1 and type-2 fuzzy inference systems with ease. Additionally, automatic membership function and rule tuning is available to you. Evaluate the created fuzzy logic systems in MATLAB and Simulink, or utilize the fuzzy inference system as a support system to explain AI-based black-box models.\cite{UserGuide}
