\section{Current overview of fuzzy modeling predator-prey game} \label{sec:scenarios}

Scientists use predator-prey models to study how animals that hunt (predators) and animals that are hunted (prey) interact in a certain environment. Fuzzy modeling is a way to study complicated things using fuzzy logic.

In a fuzzy game, the population dynamics of predators and prey are modeled using fuzzy logic. Fuzzy logic helps us understand better how predators and prey relate to each other. It does this by considering the fact that ecological systems are often uncertain and imprecise.

You can use a fuzzy inference system to model a game where a fuzzy predator chases a fuzzy prey. It has three parts: making things fuzzy, figuring out what to do, and making things clear again. Fuzzification is when we turn simple information like how many people live somewhere or things in the environment into fuzzy groups. Inference means using fuzzy rules to make predictions about how predators and prey populations will change. Defuzzification means changing the unclear results back into clear data.

Fuzzy modeling has been applied to predator-prey games in various ways to understand the behavior of the agents and to develop optimal strategies for managing populations in real-world scenarios. Here are some current overviews of fuzzy modeling predator-prey games:

Fuzzy logic-based predator-prey models: These models use fuzzy logic to develop rules for the agents' behavior based on their characteristics and the environment. The rules are defined using linguistic terms and fuzzy sets that represent the degree of membership of an agent in a particular category. These models have been used to simulate different scenarios and to evaluate the effectiveness of different strategies for managing predator and prey populations.

Fuzzy game theory-based predator-prey models: These models use fuzzy logic and game theory to analyze the behavior of the agents in predator-prey games. The models are based on the assumption that both predators and prey have incomplete information about the behavior of the other agents and are uncertain about the outcome of their actions. The models use fuzzy logic to represent the uncertainty and game theory to analyze the interactions between the agents.

Fuzzy reinforcement learning-based predator-prey models: These models use fuzzy logic and reinforcement learning to develop optimal strategies for the agents in predator-prey games. The models use fuzzy logic to represent the state of the environment and the actions of the agents and use reinforcement learning algorithms to learn the best actions to take in different situations. These models have been used to develop effective strategies for managing predator and prey populations in real-world scenarios.

Overall, fuzzy modeling can provide a more comprehensive and accurate representation of the complex dynamics between predators and prey in ecological systems. It can also be used to explore different scenarios and predict the outcomes of different management strategies.
