\section{A review of some recent modifications} \label{sec:defects}


There have been several recent modifications of fuzzy modeling predator-prey games to improve their accuracy and effectiveness. Here are some examples:

Hybrid fuzzy systems: Some researchers have combined fuzzy logic with other techniques such as neural networks or genetic algorithms to create hybrid fuzzy systems. These systems can improve the accuracy and robustness of the models by combining the strengths of different techniques. 

Multi-objective optimization: Some recent models have used multi-objective optimization to develop strategies that balance multiple objectives, such as maximizing the predator's success rate while minimizing the impact on the prey population. These models can provide more realistic and sustainable solutions to managing predator and prey populations. 

Multi objective optimization problem is the process of simultaneously optimizing two or more
conflicting objectives subject to certain constraints. Such problems can be found in various
fields: product and process design, finance, aircraft design, the oil and gas industry, automobile
design, or wherever optimal decisions need to be taken in the presence of trade‐offs between
two or more conflicting objectives. Genetic algorithms are a particular class of evolutionary
algorithms that use techniques inspired by evolutionary biology such as inheritance, mutation,
selection, and crossover and is the most commonly used search techniques in computing to find
exact or approximate solutions to such optimization and search problems.


In real world problems, parameters of a process are never precisely fixed to a definite value.
Transients, noise, measurement errors, Instrument’s least count etc makes it even more difficult
to know their exact value at any time stamp. Even if externally regulated, parameters have some
variability in their values. This variability has been continuously ignored by using
mean/approximated/fixed value of the parameters, thus losing the precious information about
the variability in the final optimized solution.

Dynamic models: Some researchers have developed dynamic fuzzy models that can adapt to changes in the environment and the behavior of the agents. These models can simulate more realistic scenarios and can be used to study the long-term effects of different management strategies.

Agent-based models: Some recent models have used agent-based modeling to simulate the behavior of individual agents in predator-prey games. These models can provide a more detailed understanding of the behavior of the agents and can be used to develop more precise management strategies.

Fuzzy decision-making: Some researchers have used fuzzy decision-making to develop strategies for managing predator and prey populations. These models use fuzzy logic to evaluate the effectiveness of different strategies and to select the best course of action based on the current state of the environment and the behavior of the agents.

Overall, these recent modifications of fuzzy modeling predator-prey games have improved the accuracy and effectiveness of the models and have provided new insights into the dynamics of predator-prey interactions. Further developments in fuzzy modeling techniques are likely to continue to improve our understanding of these complex systems and to provide more effective management strategies for predator and prey populations.
